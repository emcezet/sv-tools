\documentclass{article}

\usepackage{booktabs}
\usepackage{float}
\usepackage{tabularx}
\usepackage{amsmath}
\usepackage{amsfonts}
\usepackage{amssymb}
\usepackage{graphicx}
\usepackage{listings}
\usepackage[utf8]{inputenc}
\usepackage{indentfirst}
\usepackage{geometry}
\usepackage{syntax}
\usepackage{cite}

\let\syntleft\relax
\let\syntright\relax

\geometry{legalpaper, margin=1in}

\title{SystemVerilog Tools in Python}
\author{Michał Czyż  \\
    PG ETI UE CHIP \\
    \and 
    The Other Dude \\
    His Company / University \\
    }

\date{\today}

\begin{document}

\maketitle
\tableofcontents
\newpage

\begin{abstract}
Documentation for SystemVerilog Tools project available at:
\begin{verbatim}
    https:www.github.com/emcezet/sv-tools
\end{verbatim}

\end{abstract}

\section{Introduction}
{ The goal of project is to develop parsing engine for a subsystem of SystemVerilog (SV)
Hardware Description Language (HDL) and provide following utility:
\begin{itemize}
    \item module generation from user-defined templates,
    \item module instantiation from user file,
    \item top level file generation from user files,
    \item simple testbench file generation.
\end{itemize}
Both named port and interface style are accepted. }

\section{Order of operation.}
\begin{enumerate}
    \item Define required SystemVerilog syntax to perform given tasks.
    \item Verify if ply supports given constructs.
    \item Implement SystemVerilog lexer, parser and emitter.
\end{enumerate}

\section{Grammar coding style.}
{ Formal syntax descriptions use Backus-Naur Form (BNF). This notation was also used in Annex A of \cite{standard}.
Revised and limit list of supported constructs is provided in this section. Please note, that a specific coding
style is expected from users. Syntax notation clarification. }

\begin{enumerate}
    \item A rule is a definition (::=) of a symbol. Right hand-side of the rule is an expression.
\begin{grammar}
    <symbol> ::= __expression__
\end{grammar}
    \item Curly brackets \{ and \} denote one or more occurence of symbol.
Valid expansions of rule:
\begin{grammar}
    <A> ::= \{B\}
\end{grammar}
are $B$, $BB$, $BBBBB$, $BBB\dots B$.
    \item Square brackets [ ] denote optional symbol.
Valid expansions of rule:
\begin{grammar}
    <A> ::= [B] C
\end{grammar}
are $C$, $BC$.
    \item Keywords are denoted without any special marking.
Valid expansions of rule with keyword 'for':
\begin{grammar}
    <A> ::= [B] for C
\end{grammar}
are $for C$, $B for C$.
    \item All symbols are wrapped in $<$ $>$, except for grammar specific
    delimiters and punctuation marks, which are wrapped in ' '.
\end{enumerate}

These rules require a rewrite, what about strings and digit literals?

\section{Evaluation of ply.}
{ First task is to evaluate ply and its lexing and parsing algorithms. In particualar, optional
and multiple occurences of symbols were tested. Parsing rules tested are listed below.}

\begin{grammar}
    <A> \textbf{::=} <B>
\end{grammar}

\begin{grammar}
    <A> \textbf{::=} [C] <B>
\end{grammar}

\begin{grammar}
    <A> \textbf{::=} <B> [C]
\end{grammar}

\begin{grammar}
    <A> \textbf{::=} \{B\}
\end{grammar}

\begin{grammar}
    <A> \textbf{::=} [C] \{B\}
\end{grammar}

\begin{grammar}
    <A> \textbf{::=} \{B\} [C]
\end{grammar}

\begin{grammar}
    <A> \textbf{::=} B \alt C
\end{grammar}

Grouping of alternative, meaning it's BD or BE or CD or CE.
\begin{grammar}
    <A> \textbf{::=} (B \alt C)(D \alt E)
\end{grammar}

More complex:

\begin{grammar}
    <A> \textbf{::=} [C] <D> \{B\} [E] \{F\}
\end{grammar}


\section{Acceptable constructs in SystemVerilog.}

\subsection{Minimal formal syntax required for project.}

\begin{grammar}
    <source_text> \textbf{::=} [<timeunits_declaration>] \{<description>\}
\end{grammar}

\begin{grammar}
    <description> \textbf{::=} <module_declaration>
                               \alt <interface_declaration>
\end{grammar}

\begin{grammar}
    <module_declaration> \textbf{::=} <module_nonansi_header> [<timeunits_declaration>] \{ <module_item> \} endmodule
                                      [: <module_identifier>]
                                      \alt <module_ansi_header> [<timeunits_declaration>] \{ <non_port_module_item> \} endmodule
                                      [: <module_identifier>]
                                      \alt extern <interface_nonansi_header>
                                      \alt extern <interface_ansi_header>
\end{grammar}

\begin{grammar}
    <module_nonansi_header> \textbf{::=} \{<attribute_instance>\} <module_keyword> [<lifetime>] <module_identifier> \\
                                         \{<package_import_declaration>\} [<parameter_port_list>] <list_of_ports> ';'
\end{grammar}

\begin{grammar}
    <module_ansi_header> \textbf{::=} \{<attribute_instance>\} <module_keyword> [<lifetime>] <module_identifier> \\
                                      \{<package_import_declaration>\} [<parameter_port_list>] [<list_of_port_declarations>] ;
\end{grammar}

\begin{grammar}
    <interface_declaration> \textbf{::=} <interface_nonansi_header> [<timeunits_declaration>] \{<interface_item>\} \\
                                         <endinterface> [':' <interface_identifier>]
                                         \alt <interface_ansi_header> [<timeunits_declaration>] \{<non_port_interface_item>\} \\
                                         <endinterface> [':' <interface_identifier>]
                                         \alt \{<attribute_instance>\} interface <interface_identifier> ( .* ) ;
                                         [<timeunits_declaration>] \{<interface_item>\} endinterface [':' <interface_identifier>]
                                         \alt extern <interface_nonansi_header>
                                         \alt extern <interface_ansi_header>
\end{grammar}

\begin{grammar}
<interface_nonansi_header> \textbf{::=} \{<attribute_instance>\} interface [<lifetime>] <interface_identifier> \\
                                        \{<package_import_declaration>\} [<parameter_port_list>] <list_of_ports> ';'
\end{grammar}
\begin{grammar}
<interface_ansi_header> \textbf{::=} \{<attribute_instance>\} interface [<lifetime>] <interface_identifier> \\
                                     \{<package_import_declaration>\} [<parameter_port_list>] [<list_of_port_declarations>] ';'
\end{grammar}


{ Please note, that two original rules conflict in context-less grammars. Is a standalone 'x' a digit or an identifier?
Same happens for z_digit and exp. Currently[ ] denotes regexp, not optional, that needs a fix!

\begin{grammar}
    <x_digit> \textbf{::=} x | X
\end{grammar}

%\begin{grammar}
    %simple_identifier \textbf{::=} [ a-zA-Z_ ] \{ [ a-zA-Z0-9_\$ ] \}
%\end{grammar}

{ This requires a rewrite to following set of rules. }

\begin{grammar}
    <number> \textbf{::=} <integral_number>
                          \alt <real_number>
\end{grammar}

\begin{grammar}
<integral_number> \textbf{::=} <decimal_number>
                               \alt <octal_number>
                               \alt <binary_number>
                               \alt <hex_number>
\end{grammar}

\begin{grammar}
<decimal_number> \textbf{::=} <unsigned_number>
                              \alt [ size ] decimal_base unsigned_number
                              \alt [ size ] decimal_base x_digit { _ }
                              \alt [ size ] decimal_base z_digit { _ }
\end{grammar}

\begin{grammar}
    <binary_number> \textbf{::=} [<size>] <binary_base> <binary_value>
\end{grammar}

\begin{grammar}
    <octal_number> \textbf{::=} [<size>] <octal_base> <octal_value>
\end{grammar}

\begin{grammar}
    <hex_number> \textbf{::=} [<size>] <hex_base> <hex_value>
\end{grammar}

\begin{grammar}
    <sign> \textbf{::=} + \alt -
\end{grammar}

\begin{grammar}
    <size> \textbf{::=} <non_zero_unsigned_number>
\end{grammar}

\begin{grammar}
    <non_zero_unsigned_number> \textbf{::=} <non_zero_decimal_digit> { _ \alt <decimal_digit>}
\end{grammar}


\begin{grammar}
    <real_number> \textbf{::=} fixed_point_number
                               \alt unsigned_number [ . unsigned_number ] exp [ sign ] unsigned_number

\end{grammar}

\begin{grammar}
    <fixed_point_number33> \textbf{::=} unsigned_number \. unsigned_number
\end{grammar}

\begin{grammar}
    <exp> \textbf{::=} [eE]
\end{grammar}

\begin{grammar}
    <unsigned_number> \textbf{::=} <decimal_digit> \{ _ \alt <decimal_digit> \}
\end{grammar}

\begin{grammar}
    <binary_base_value> \textbf{::=} ['] [sS] [bB] ([xX] \alt [zZ?] \alt [0-1]) \{ _ \alt ([xX] \alt [zZ?] \alt [0-1]) \}
\end{grammar}

\begin{grammar}
    <octal_value> \textbf{::=} ['][sS][oO] ([xX] \alt [zZ?] \alt [0-7]) \{ _ | ([xX] \alt [zZ?] \alt [0-7]) \}
\end{grammar}

\begin{grammar}
    <hex_value> \textbf{::=} ['][sS][hH] ([xX] \alt [zZ?] \alt [0-9] \alt [a-fA-F]) \{ _ | ([xX] \alt [zZ?] \alt [0-9] \alt [a-fA-F]) \}
\end{grammar}|

\begin{grammar}
    <decimal_base> \textbf{::=} ['][sS][dD]
\end{grammar}

\begin{grammar}
    <non_zero_decimal_digit> \textbf{::=} [1-9]
\end{grammar}

\begin{grammar}
    <decimal_digit> \textbf{::=} [0-9]
\end{grammar}

\begin{grammar}
    <unbased_unsized_literal> \textbf{::=} ['][0-1] | ['][xX] | ['][zZ?]
\end{grammar}


%\lstinputlisting[language = verilog]{../SVEngine/SVExamples/basic_module.sv}


\section{Usage.}
{ All scripts support -h and --help parameters for usage help.}

\begin{thebibliography}{9}
\bibitem{standard}
8299595,IEEE Std 1800-2017 (Revision of IEEE Std 1800-2012),IEEE Standard for SystemVerilog--Unified Hardware Design, Specification, and Verification Language,
2018,Feb
\bibitem{placeholder}
placeholder

\end{thebibliography}

\end{document}
