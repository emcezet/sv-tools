\documentclass{article}

\usepackage{booktabs}
\usepackage{float}
\usepackage{tabularx}
\usepackage{amsmath}
\usepackage{amsfonts}
\usepackage{amssymb}
\usepackage{graphicx}
\usepackage{listings}
\usepackage[utf8]{inputenc}
\usepackage{indentfirst}
\usepackage{geometry}
\geometry{legalpaper, margin=1in}

\title{SystemVerilog Tools in Python}
\author{Michał Czyż  \\
    PG ETI UE CHIP \\
    \and 
    The Other Dude \\
    His Company / University \\
    }

\date{\today}

\begin{document}

\maketitle
\tableofcontents
\newpage

\begin{abstract}
Documentation for SystemVerilog Tools project available at:
\begin{verbatim}
    https:www.github.com/emcezet/sv-tools
\end{verbatim}

\end{abstract}

\section{Introduction}
{ The goal of project is to develop a SystemVerilog (SV) parsing engine and
provide scripts for simple module generation, instantiation in both
named ports and interface modports styles. }

\section{Acceptable constructs in SystemVerilog.}

\subsection{Module declaration.}
{ The basic building block of SystemVerilog language is a module. Any SV module
consists of a name, a list of parameters and a list of ports. An example is
provided below: }

\lstinputlisting[language = verilog]{../SVEngine/SVExamples/basic_module.sv}

{ Keywords \textbf{module} and \textbf{endmodule} are always at the start and
 the end of the module. After the keyword \textbf{module} a module name is given.
 It is optional to end the module declaration with \textbf{endmodule : module\_name}.
 The \textbf{\#} sign marks the beginning of list of parameters, which is comma separated.
 List of parameters is optional. An empty list is also accepted.}

{ Port list is also a comma separated list of entries. Each port requires direction
and name. There are three supported directions: \textbf{input}, \textbf{output},
\textbf{inout}. Bus signals have width encoded in square brackets. Interfaces are
used via modports. The declaration is \textbf{interface\_name.modport\_name port\_name.} }

{ SV supports C-style comments and macros. The only difference is that used macro
requires a tick. Whitespaces are ignored.}

\subsection{Module instantiation.}

\subsection{Interface declaration.}

\section{Usage.}
{ All scripts support -h and --help parameters for usage help.}

\end{document}
