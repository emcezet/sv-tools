\documentclass{article}

\usepackage{booktabs}
\usepackage{float}
\usepackage{tabularx}
\usepackage{amsmath}
\usepackage{amsfonts}
\usepackage{amssymb}
\usepackage{graphicx}
\usepackage{listings}
\usepackage[utf8]{inputenc}
\usepackage{indentfirst}
\usepackage{geometry}
\usepackage{syntax}
\usepackage{cite}

\geometry{legalpaper, margin=1in}

\title{SystemVerilog Tools in Python}
\author{Michał Czyż  \\
    PG ETI UE CHIP \\
    \and 
    The Other Dude \\
    His Company / University \\
    }

\date{\today}

\begin{document}

\maketitle
\tableofcontents
\newpage

\begin{abstract}
Documentation for SystemVerilog Tools project available at:
\begin{verbatim}
    https:www.github.com/emcezet/sv-tools
\end{verbatim}

\end{abstract}

\section{Introduction}
{ The goal of project is to develop parsing engine for a subsystem of SystemVerilog (SV)
Hardware Description Language (HDL) and provide following utility:
\begin{itemize}
    \item module generation from user-defined templates,
    \item module instantiation from user file,
    \item top level file generation from user files,
    \item simple testbench file generation.
\end{itemize}
Both named port and interface style are accepted. }

\section{Grammar coding style.}
{ Formal syntax descriptions use Backus-Naur Form (BNF). This notation was also used in Annex A of \cite{standard}.
Revised and limit list of supported constructs is provided in this section. Please note, that a specific coding
style is expected from users. Syntax notation clarification.}

\begin{enumerate}
    \item Every rule consists of symbols. Right hand-side of rule is called an expression.
    \item Curly brackets \{ and \} denote one or more occurence of symbol.
    \item Square brackets [ ] denote optional symbol.
    \item Keywords are denoted without any special marking.
    \item All other symbols are wrapped in $<$ $>$.
    \item Grammar specific delimiters and punctuation marks are wrapped in ' '.
\end{enumerate}

\begin{grammar}
    <symbol> ::= __expression__
\end{grammar}

\section{Evaluation of ply.}
{ First task is to evaluate ply and its lexing and parsing algorithms. In particualar, optional
and multiple occurences of symbols were tested. Parsing rules tested are listed below.}

\begin{grammar}
    <A> \textbf{::=} <B>
\end{grammar}

\begin{grammar}
    <A> \textbf{::=} [C] <B>
\end{grammar}

\begin{grammar}
    <A> \textbf{::=} <B> [C]
\end{grammar}

\begin{grammar}
    <A> \textbf{::=} \{B\}
\end{grammar}

\begin{grammar}
    <A> \textbf{::=} [C] \{B\}
\end{grammar}

\begin{grammar}
    <A> \textbf{::=} \{B\} [C]
\end{grammar}

More complex:

\begin{grammar}
    <A> \textbf{::=} [C] <D> \{B\} [E] \{F\}
\end{grammar}


\section{Acceptable constructs in SystemVerilog.}

\subsection{Formal syntax.}

\begin{grammar}
    <source_text> \textbf{::=} [<timeunits_declaration>] \{<description>\}
\end{grammar}

\begin{grammar}
    <description> \textbf{::=} <module_declaration>
                               \alt <interface_declaration>
\end{grammar}

\begin{grammar}
    <module_declaration> \textbf{::=} <module_nonansi_header> [<timeunits_declaration>] \{ <module_item> \} endmodule
                                      [: <module_identifier>]
                                      \alt <module_ansi_header> [<timeunits_declaration>] \{ <non_port_module_item> \} endmodule
                                      [: module_identifier]
                                      \alt extern <interface_nonansi_header>
                                      \alt extern <interface_ansi_header>
\end{grammar}

\begin{grammar}
    <module_nonansi_header> \textbf{::=} \{<attribute_instance>\} <module_keyword> [<lifetime>] <module_identifier> \\
                                         \{<package_import_declaration>\} [<parameter_port_list>] <list_of_ports> ';'
\end{grammar}

\begin{grammar}
    <module_ansi_header> \textbf{::=} \{<attribute_instance>\} <module_keyword> [<lifetime>] <module_identifier> \\
                                      \{<package_import_declaration>\} [<parameter_port_list>] [<list_of_port_declarations>] ;
\end{grammar}

\begin{grammar}
    <interface_declaration> \textbf{::=} <interface_nonansi_header> [<timeunits_declaration>] \{<interface_item>\} \\
                                         <endinterface> [':' <interface_identifier>]
                                         \alt <interface_ansi_header> [<timeunits_declaration>] \{<non_port_interface_item>\} \\
                                         <endinterface> [':' <interface_identifier>]
                                         \alt \{<attribute_instance>\} interface <interface_identifier> ( .* ) ;
                                         [<timeunits_declaration>] \{<interface_item>\} endinterface [':' <interface_identifier>]
                                         \alt extern <interface_nonansi_header>
                                         \alt extern <interface_ansi_header>
\end{grammar}

\begin{grammar}
<interface_nonansi_header> \textbf{::=} \{<attribute_instance>\} interface [<lifetime>] <interface_identifier> \\
                                        \{<package_import_declaration>\} [<parameter_port_list>] <list_of_ports> ';'
\end{grammar}
\begin{grammar}
<interface_ansi_header> \textbf{::=} \{<attribute_instance>\} interface [<lifetime>] <interface_identifier> \\
                                     \{<package_import_declaration>\} [<parameter_port_list>] [<list_of_port_declarations>] ';'
\end{grammar}





%\lstinputlisting[language = verilog]{../SVEngine/SVExamples/basic_module.sv}


\section{Usage.}
{ All scripts support -h and --help parameters for usage help.}

\begin{thebibliography}{9}
\bibitem{standard}
8299595,IEEE Std 1800-2017 (Revision of IEEE Std 1800-2012),IEEE Standard for SystemVerilog--Unified Hardware Design, Specification, and Verification Language,
2018,Feb
\bibitem{placeholder}
placeholder

\end{thebibliography}

\end{document}

